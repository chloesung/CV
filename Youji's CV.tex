
\documentclass{resume} % Use the custom resume.cls style

\usepackage[left=0.75in,top=0.6in,right=0.75in,bottom=0.6in]{geometry} % Document margins
\usepackage{multicol}
\setlength{\columnsep}{1cm}

\name{Youji Sung}
\address{010 4242 7565 \\ chloesung@korea.ac.kr \\ github.com/chloesung}% Your name
\begin{document}

%----------------------------------------------------------------------------------------
%	EDUCATION SECTION
%----------------------------------------------------------------------------------------

\begin{rSection}{Education}
{\bf Korea University} \hfill {\em 2018-Present} \\ 
College of Political Science and Economics\\
Department of Economics\smallskip \\\
GPA: 3.96 / 4.5

{\bf Hanyoung Foreign Langauage High school} \hfill {\em 2015-2017}\\
Major in Chinese\smallskip
\end{rSection}
%----------------------------------------------------------------------------------------
%	Relevent course
%----------------------------------------------------------------------------------------
\begin{rSection}{Relevant Courseworks}
Econometrics, Game Theory, Seminar on Money \& Banking, Corporate Finance,\\ Calculus, Linear Algebra, Real Analysis, Statistical Computer Software, Computer Language
\end{rSection}

%----------------------------------------------------------------------------------------
%	WORK EXPERIENCE SECTION
%----------------------------------------------------------------------------------------

\begin{rSection}{Experience}
\begin{rSubsection}{NH Investor Profiling Competition}{November 2020 - Present}{Analyzing trading data to profile investment behavior of 2,30's}{NH Investment Bank, Seoul}
\item analyzed and visualized the trading data with Python
\item found insights from the data and made suggestions for NH Investment Bank
\end{rSubsection}

%------------------------------------------------

\begin{rSubsection}{Financial Engineering Research Competition \\ - The First Prize}{January 2021}{Option Value Calculating and Visualizing Package for Python}{Korea University, Seoul}
\item wrote codes to implement option valuation models in Python
\item made user guides and example codes for the package
\end{rSubsection}

%------------------------------------------------

\begin{rSubsection}{Jinri Scholarship}{May - December 2020}{Economic Research: Reasons of Youth Unemployment}{Korea University, Seoul}
\item designed the youth unemployment model for our hypothesis test
\item gathered and analyzed the wage data with Python and R
\end{rSubsection}
\end{rSection}
%------------------------------------------------


%----------------------------------------------------------------------------------------
%	TECHNICAL STRENGTHS SECTION
%----------------------------------------------------------------------------------------

\begin{rSection}{Technical Strengths}

\begin{tabular}{ @{} >{\bfseries}l @{\hspace{6ex}} l }
Programming & Python, R, C++, SAS \\
Data Science &  Python with Numpy and Pandas, R\\


\end{tabular}

\end{rSection}

%	Volunteering Activities and Projects
%----------------------------------------------------------------------------------------
\begin{rSection}{Activities}
\begin{rSubsection}{KUBIG}{2020-Present}{Staff}{Seoul}
\item data analysis club in department of statistics, Korea University
\item studied about machine learning and data analysis with Python
\item worked on projects based on machine learning

\end{rSubsection}
%------------------------------------------------
\begin{rSubsection}{Korea college student Mentor Union}{2019-Present}{Mentor and Staff}{Seoul}
\item gave a lecture about Economics to the middle school and high school students
\item as a staff, planned and managed the multiple events for the union

\end{rSubsection}
\end{rSection}
%------------------------------------------------

\end{document}
